\documentclass{article}
\usepackage[utf8]{inputenc}
%\usepackage{graphicx}

\usepackage[parfill]{parskip}    % Activate to begin paragraphs with an empty line rather than an indent
%\usepackage{graphicx}
%\usepackage{amssymb}
%\usepackage{epstopdf}

%\DeclareGraphicsRule{.tif}{png}{.png}{`convert #1 `dirname #1`/`basename #1 .tif`.png}
%\usepackage{graphicx}
%\usepackage{xcolor}


\usepackage{AVGMacros}
%\usepackage{ImageMacros}
%=====================================================
%
%       BildMacros.tex
%
%======================================================
%
%  These are macros which are related to representing figures in LaTeX documents
%  in a simple and standardized way.
%   The explanation texts are still largely in German (sorry for that!),
%   but will be translated as soon as possible.
%
%  Source: Rudolf Mester
%=========================================================

%---------------------------------------------------------------------------
%Dependencies
\usepackage{tcolorbox}

%---------------------------------------------------------------------------
%Variables
\newlength{\CVGMBildEins}
\newlength{\CVGMBildZwei}
\newlength{\CVGMBildTotal}
\newlength{\CVGMMinipageEins}
\newlength{\CVGMMinipageZwei}

%---------------------------------------------------------------------------
%   SECTION 1:  commands for text mode
%---------------------------------------------------------------------------
%
% ----- Bilder
%
% Mit den folgenden beiden Befehlen wird eingestellt, ob Breitenangaben in
% den Bildbefehlen absolut (z.B. "7.5cm") oder relativ zur \linewidth
% interpretiert werden (z.B. "0.3"; \linewidth wird automatisch erg\"{a}nzt).
% Voreingestellt ist der absolute Modus.
%
% Selbstverst\"{a}ndlich ist es noch m\"{o}glich, im Absolutmodus durch Angabe
% von \linewidth im Aufruf eine relative Angabe zu machen; durch die
% automatische Erg\"{a}nzung ist es dagegen im Relativmodus nicht m\"{o}glich,
% absolute Angaben zu machen; erst ist der Modus zu wechseln.
%
\newcommand{\absolutemeasures}{\gdef\cvgmode{}}
\newcommand{\relativemeasures}{\gdef\cvgmode{\linewidth}}
\absolutemeasures % default
%
% Grunds\"{a}tzliches:
%
% F\"{u}r 1 oder 2 Bilder gibt es je drei Befehle:
%  - \(doppel)bild : Bild(er) mit Bildunterschrift(en) und Label
%  - \(doppel)bildnocap : ohne Unterschriften, als Flie{\ss}objekt
%  - \(doppel)bildnofloat : keine Unterschriften, kein Flie{\ss}objekt;
%    Bilder "hier und jetzt" im Text, evtl. mit "underfull vbox"
%    auf der n\"{a}chsten Seite ganz oben
%
% Zus\"{a}tzlich gibt es einen Befehl f\"{u}r zwei Bilder mit gemeinsamer
% Bildunterschrift: \doppelbildonecap
%
% Weiterhin gibt es noch Makros f\"{u}r ein Bild mit Text daneben, und zwar
% entweder links Bild und rechts Text oder umgekehrt.
%
%
% Schalter:
%
% Es gibt zwei Schalter (boolean Variablen), die das Verhalten steuern:
%  - \showfilenamestrue  bzw.  \showfilenamesfalse  gibt an, ob hinter
%    der Bildunterschrift der Dateiname erscheinen soll. Dies kann f\"{u}r
%    Debugging-Zwecke hilfreich sein.
%  - \emptypicstrue  bzw.  \emptypicsfalse  dient dazu, noch nicht
%    exitierende Zeichnungen schon einzubinden mit allen Referenzen.
%    Der Dateiname erscheint in einem Kasten (H\"{o}he = 3/4 der angegebenen
%    Bildbreite). Anstelle des Dateinamens kann auch eine kurze
%    Bildbeschreibung angegeben werden, die dann anstelle des Dateinamens
%    angezeigt wird. Beide Schalter gleichzeitig einzuschalten macht
%    aber dann nat\"{u}rlich nicht mehr viel Sinn.
%
\newif\ifshowfilenames\showfilenamesfalse
\def\showfilename#1{\ifshowfilenames\fbox{\textbf{ #1}}\fi}
\newif\ifemptypics\emptypicsfalse
\def\myincludegraphics#1#2{%
    \ifemptypics%
        \@tempdima=#2\cvgmode%
        \fbox{\parbox[b][.75\@tempdima][c]{\@tempdima}{#1}}%
    \else%
        \if@noscale%
	    \includegraphics{#1}%
	\else%
        \includegraphics[width=#2\cvgmode]{#1}%
	\fi%
    \fi%
}
\def\MYincludegraphics#1{%
    \ifemptypics%
        \fbox{\parbox[b][.75\linewidth][c]{\linewidth}{#1}}%
    \else%
        \if@noscale%
        \includegraphics{#1}%
	\else%
        \includegraphics[width=\linewidth]{#1}%
	\fi%
    \fi%
}

\newcommand{\image}[4]
{
    \begin{figure}
        \begin{center}
            \myincludegraphics{#1}{#2}
            \small\caption{\label{#4}#3\showfilename{#1}}
        \end{center}
    \end{figure}
}

%----------------- new command: \bildsc ------------------------------------
%
% Makro zum Anzeigen eines Bildes
% \"{U}bergabeparameter (4): 1: Filename
%                        2: Breite des Bildes
%                        3: kurze Bildunterschrift (Abbildungsverz.)
%                        3: Bildunterschrift
%                        4: Label f\"{u}r Bild
%
\newcommand{\imagesc}[5]
{
  \begin{figure}
      \begin{center}
          \myincludegraphics{#1}{#2}
          \small\caption[#3]{\label{#5}#4\showfilename{#1}}
      \end{center}
  \end{figure}
}

%----------------- new command: \bildnocap ---------------------------------
%
% Bild ohne Bildunterschrift (caption)
% \"{U}bergabeparameter (2): 1: Filename
%                        2: Breite des Bildes
%
\newcommand{\imagenocap}[2]{
    \begin{figure}
        \begin{center}
            \myincludegraphics{#1}{#2}
        \end{center}
        \hspace{0pt plus 1fill}\showfilename{#1}\hspace{0pt plus 1fill}
    \end{figure}
}

% das gleiche als nicht-Flie{\ss}objekt; logischerweise ohne Bildunterschrift
\newcommand{\imagenofloat}[2]{
    \begin{center}
        \myincludegraphics{#1}{#2}
    \end{center}
    \hspace{0pt plus 1fill}\showfilename{#1}\hspace{0pt plus 1fill}
}

%----------------- new command: \doppelbild --------------------------------
%Not working, no CVGMBILDEins, CVGMBILDZwei or CVGBILDTotal
%
% 2 Bilder nebeneinander
% \"{U}bergabeparameter (8): 1: Filename erstes Bild
%                        2: Breite des ersten Bildes
%                        3: Bildunterschrift erstes Bild
%                        4: Label f\"{u}r erstes Bild
%                        5-8: genauso f\"{u}r zweites Bild
%
\newcommand{\doubleimage}[8]{
    \if@noscale
	      % Obtain and save the native width of each picture in the length definitions.
	      \settowidth{\CVGMBildEins}{\MYincludegraphics{#1}}
	      \settowidth{\CVGMBildZwei}{\MYincludegraphics{#1}}
	      \setlength{\CVGMBildTotal}{\CVGMBildEins + \CVGMBildZwei}
        \setlength{\CVGMMinipageEins}{\linewidth * \real{0.95} * \ratio{\CVGMBildEins}{\CVGMBildTotal}}
	      \setlength{\CVGMMinipageZwei}{\linewidth * \real{0.95} * \ratio{\CVGMBildZwei}{\CVGMBildTotal}}        
    \else
	      \setlength{\CVGMMinipageEins}{#2\cvgmode}
	      \setlength{\CVGMMinipageZwei}{#6\cvgmode}
    \fi
    \begin{figure}
        \begin{center}
            \begin{minipage}[t]{\CVGMMinipageEins}
                \if@noscale
                    \centering
                \fi
                \MYincludegraphics{#1}
                \small\caption{\label{#4}#3\showfilename{#1}}
            \end{minipage}
            \hfill
            \begin{minipage}[t]{\CVGMMinipageZwei}
                \if@noscale
                    \centering
                \fi
                \MYincludegraphics{#5}
                \small\caption{\label{#8}#7\showfilename{#5}}
            \end{minipage}
        \end{center}
    \end{figure}
}

%--------------------- new command: \dimage --------------------------------
% Used as an image object for the double image functions below
%
% 1 Bild zur Verwendung mit dem \doppelbildparts Befehl
% \"{U}bergabeparameter (4): 1: Filename Bild
%                        2: Breite des Bildes
%                        3: Bildunterschrift Bild
%                        4: Label f\"{u}r Bild
%

\newcommand{\dimage}[4]{
            \begin{minipage}[t]{#2\cvgmode}
                \MYincludegraphics{#1}
                \small\caption{\label{#4}#3\showfilename{#1}}
            \end{minipage}
}

%--------------------- new command: \dbildsc ------------------------------
%
% 1 Bild zur Verwendung mit dem \doppelbildparts Befehl
% \"{U}bergabeparameter (5): 1: Filename Bild
%                        2: Breite des Bildes
%                        3: kurze Bildunterschrift (Abbildungsverz.)
%                        4: Bildunterschrift Bild
%                        5: Label f\"{u}r Bild
%
\newcommand{\dimagesc}[5]{
            \begin{minipage}[t]{#2\cvgmode}
                \MYincludegraphics{#1}
                \small\caption[#3]{\label{#5}#4\showfilename{#1}}
            \end{minipage}
}

%----------------- new command: \doppelbild --------------------------------
%
% 2 Bilder nebeneinander
% \"{U}bergabeparameter (2): 1: \dbild oder \dbildsc (linkes Bild)
%                        2: ebenso (rechtes Bild)
%
\newcommand{\doubleimageparts}[2]{
    \begin{figure}
        \begin{center}
            {#1}
            \hfill
            {#2}
        \end{center}
    \end{figure}
}

%----------------- new command: \doppelbildnocap ---------------------------
%
% 2 Bilder nebeneinander, ohne \"{U}berschrift
% \"{U}bergabeparameter (4): 1: Filename erstes Bild
%                        2: Breite des ersten Bildes
%                        3-4: genauso f\"{u}r zweites Bild
% Der Platz zwischen den beiden Bildern ist durch den Befehl
% \doppelbildsep bestimmt.
% default = \hfill, d.h. auf volle Breite gestreckt; ggf. umdefinieren,
% z.B. mit \renewcommand{\doppelbildsep}{\hspace{.1\linewidth}}, dann
% automatisch zentriert.
\newcommand{\doppelbildsep}{\hfill}
%
\newcommand{\doubleimagenocap}[4]{
    \begin{figure}
        \begin{center}
            \myincludegraphics{#1}{#2}%
            \doppelbildsep%
            \myincludegraphics{#3}{#4}%
            \hspace{0pt plus 1fill}\showfilename{#1}%
            \hspace{0pt plus 2fill}\showfilename{#3}\hspace{0pt plus 1fill}
        \end{center}
    \end{figure}
}

% das gleiche als nicht-Flie{\ss}objekt; logischerweise ohne Bildunterschriften
%
\newcommand{\doubleimagenofloat}[4]{
    \begin{center}
        \myincludegraphics{#1}{#2}
        \doppelbildsep
        \myincludegraphics{#3}{#4}
    \end{center}
    \hspace{0pt plus 1fill}\showfilename{#1}%
    \hspace{0pt plus 2fill}\showfilename{#3}\hspace{0pt plus 1fill}
}


%----------------- new command: \doppelbildonecap -----------------------
%
% 2 Bilder nebeneinander, EINE Unterschrift
% \"{U}bergabeparameter: 1: Filename erstes Bild
%                    2: Breite des ersten Bildes
%                    3: Filename des zweiten Bildes
%                    4: Breite des zweiten Bildes
%                    5: Bildunterschrift
%                    6: Label
%
\newcommand{\doubleimageonecap}[6]{
    \begin{figure}
        \begin{center}
            \myincludegraphics{#1}{#2}
            \doppelbildsep
            \myincludegraphics{#3}{#4}
            \small\caption{\label{#6}#5\showfilename{#1}\showfilename{#3}}
        \end{center}
    \end{figure}
}

%----------------- new command: \doppelbildonecapsc -----------------------
%
% 2 Bilder nebeneinander, EINE Unterschrift
% \"{U}bergabeparameter: 1: Filename erstes Bild
%                    2: Breite des ersten Bildes
%                    3: Filename des zweiten Bildes
%                    4: Breite des zweiten Bildes
%                    5: kurze Bildunterschrift (Abbildungsverz.)
%                    6: Bildunterschrift
%                    7: Label
%
\newcommand{\doubleimageonecapsc}[7]{
    \begin{figure}
        \begin{center}
            \myincludegraphics{#1}{#2}
            \hfill
            \myincludegraphics{#3}{#4}
            \small\caption[#5]{\label{#7}#6\showfilename{#1}\showfilename{#3}}
        \end{center}
    \end{figure}
}


%%%%%%%%%%%%%%%%%% Bild links, Text rechts
%
% [#1]: Gesamtbreite; OPTIONAL!
% #2: Dateiname Bild
% #3: Breite Bild
% #4: Text
% #5: Breite Text
% Beispiel: \BildUndText[.9]{goethe.eps}{.25}{blablablubb}{.6}
%
\newcommand{\ImageAndText}[5][\linewidth]{
  \begin{center}
    \makebox[#1]{
      \if@noscale
	        % Obtain and save the native width of each picture in the length definitions.
	        \settowidth{\CVGMBildEins}{\myincludegraphics{#2}{#3}}
	        \settowidth{\CVGMBildZwei}{\qquad\quad}
          \setlength{\CVGMMinipageEins}{\linewidth - \CVGMBildZwei - \CVGMBildEins}
      \else
	        \setlength{\CVGMMinipageEins}{#5\cvgmode}
      \fi

      \raisebox{-\totalheight}{\myincludegraphics{#2}{#3}}
      \hfill
      \raisebox{-\totalheight}{\begin{minipage}[b]{\CVGMMinipageEins}{#4}
      \end{minipage}}
    }
    \showfilename{#2}
  \end{center}
}

%%%%%%%%%%%%%%%%%% Text links, Bild rechts
%
% [#1]: Gesamtbreite; OPTIONAL!
% #2: Text
% #3: Breite Text
% #4: Dateiname Bild
% #5: Breite Bild
%
\newcommand{\TextAndImage}[5][\linewidth]{
  \begin{center}
    \makebox[#1]{

      \if@noscale
	        % Obtain and save the native width of each picture in the length definitions.
	        \settowidth{\CVGMBildEins}{\myincludegraphics{#4}{#5}}
	        \settowidth{\CVGMBildZwei}{\qquad\quad}
          \setlength{\CVGMMinipageEins}{\linewidth - \CVGMBildZwei - \CVGMBildEins}
      \else
	        \setlength{\CVGMMinipageEins}{#3\cvgmode}
      \fi

      \raisebox{-\totalheight}{\begin{minipage}[b]{\CVGMMinipageEins}{#2}
      \end{minipage}}
      \hfill
      \raisebox{-\totalheight}{\myincludegraphics{#4}{#5}}
    }
    \showfilename{#4}
  \end{center}
}



\title{Some Examples for the AROS Vision Group Latex Macros (AVGMacro.sty)}
\author{Andreas L. Teigen \and Rudolf Mester }
\date{October 2020}

\begin{document}

\maketitle

\section{Introduction}

\subsection{Person-specific comments / notes}

\verb"\note{This is an anonymous test note}"\\
yields:\\
\note{This is an anonymous test note}

\verb"\rmnote{This is a note by Rudolf}"\\
yields:\\
\rmnote{This is a note by Rudolf}

Similarly:
\asnote{This is a note by Annette}
\atnote{This is a note by Andreas}
\mynote{This is a note by Mauhing}
\pznote{This is a note by Peder}
\ahnote{This is a note by Axel}
\wknote{This is a note by William}

\subsection{Macros for marking up text parts for editing}

\bit
\item
\verb"\deletetext{This text should be deleted}"\\
yields\\
\deletetext{This text should be deleted}

\item
\verb"\replacetext{This text should be}{changed into this}"\\
yields\\
\replacetext{This text should be}{changed into this}

\item
\verb"\inserttext{This text should be inserted}"\\
yields:\\
\inserttext{This text should be inserted}

\item
\verb"\missingtext"\\
yields:\\
\missingtext \qquad \qquad(This is a symbol that here some text should be inserted, perhaps
a word missing, etc.)

\item
\verb"\markuptext{This text is highlighted}"\\
yields:\\
\markuptext{This text is highlighted}

\item
The macro \verb"infosource{...}" allows the mark the source of some information
in a text in a standardized way. Example:
\verb"infosource{NTNU home page}"\\
yields
\infosource{NTNU home page}
\eit

\section{Mathematical macros}

\subsection{Environments for equations in display style}

Writing \verb"\begin{equation} ... \end{equation}" etc.\ is often tedious.
Here are some shortcuts:
\bit
\item \verb"\begin{equation} ... \end{equation}" $ \quad \longrightarrow \quad$
\verb"\beq ... \eeq"

\item \verb"\begin{displaymode} ... \end{displaymode}" 
$ \quad \longrightarrow \quad$
\verb"\bdm ... \edm"


\item \verb"\begin{itemize} ... \end{itemize}" 
$ \quad \longrightarrow \quad$
\verb"\bit ... \eit"


\item \verb"\begin{enumerate} ... \end{enumerate}" 
$ \quad \longrightarrow \quad$
\verb"\benum ... \eenum"

\eit

%\beq 
%x = \frac{1+e^y}{1-e^y{+1}}\\
%x = \frac{1+e^y}{1-e^y{+1}}
%\eeq
%
%\bdm
%x = \frac{1+e^y}{1-e^y{+1}}\\
%x = \frac{1+e^y}{1-e^y{+1}}
%\edm

\bit
\item Equation arrays:\\
\begin{verbatim}
\bea x &=& \frac{1+e^y}{1-e^y{+1}}\\
\frac{1+e^y}{1-e^y{+1}} &=& x
\eea
\end{verbatim}

yields
\bea x &=& \frac{1+e^y}{1-e^y{+1}}\\
\frac{1+e^y}{1-e^y{+1}} &=& x
\eea
\eit

%\bit
%\item first
%\item second
%\item third
%\eit 
%
%\benum
%\item first
%\item second
%\item third
%\eenum 


\begin{table}[h]
\begin{tabular}{lll}
\verb"$a \yields b$"          & $\quad \longrightarrow \quad$ & $a \yields b$          \\
\verb"$a \shallbe b$"         & $\quad \longrightarrow \quad$ & $a \shallbe b$         \\
\verb"$a \definedas b$"       & $\quad \longrightarrow \quad$ & $a \definedas b$       \\
\verb"$a \isapproximately b$" & $\quad \longrightarrow \quad$ & $a \isapproximately b$ \\
\verb"$\Prob{x\given y}$"     & $\quad \longrightarrow \quad$ & $\Prob{x\given y}$     \\
\verb"$\Erw{X}$"              & $\quad \longrightarrow \quad$ & $\Erw{X}$              \\
\verb"$\Var{X}$"              & $\quad \longrightarrow \quad$ & $\Var{X}$              \\
\verb"$\Cov{X, Y}$"           & $\quad \longrightarrow \quad$ & $\Cov{X, Y}$          \\
\verb"$\pdf{f(x)}$"              & $\quad \longrightarrow \quad$ & $\pdf{f(x)}$              \\
\verb"$\Cor{X,Y}$"              & $\quad \longrightarrow \quad$ & $\Cor{X,Y}$         
\end{tabular}
\end{table}

\subsection{Vectors and Matrices}

\subsubsection{Vectors}

There are predefined macros for all small letters that denote a vector:\\
\verb"$\va = \vb + \vc$"               $\quad \longrightarrow \quad  \va = \vb + \vc$\\
Note that by redefinition of a \emph{single} macro, vector notation
can be changed from a vector symbol above the character
to a boldface character.

The \verb"\Vector{...}" macro simplifies the task of typesetting column vectors
with explicit display of the vector elements:

\verb"$ \va = \Vector{1 \\ 2 \\ 3 \\4}$"  yields
\bdm
 \va =  \Vector{1 \\ 2 \\ 3 \\4}
\edm


\subsubsection{Matrices}

Like in most books, matrices are typeset, by convention, as boldface uppercase letters:\\
\verb"$\mat{X}$"               $\quad \longrightarrow \quad \mat{X}$\\
However, this macro is rarely directly used.
There are predefined macros for all capital letters that denote a matrix:\\
\verb"$\MA = \MB^{-1} \cdot \MC$"               $\quad \longrightarrow \quad  \MA = \MB^{-1} \cdot \MC$ 


\subsubsection{Further macros on vectors and matrices}

\begin{table}[h]
\begin{tabular}{lll}
\verb"$\det{\MB}$"              & $\quad \longrightarrow \quad$ & $\det{\MB}$              \\
\verb"$\trace{\MA}$"              & $\quad \longrightarrow \quad$ & $\trace{\MA}$              \\
\verb"$\rank{\MA}$"              & $\quad \longrightarrow \quad$ & $\rank{\MA}$              \\
\verb"$\diag{a_i}$"              & $\quad \longrightarrow \quad$ & $\diag{a_i}$              \\
\verb"$\abs{a-b}$"              & $\quad \longrightarrow \quad$ & $\abs{a-b}$              \\
\verb"$\norm{\MA-\MB}$"              & $\quad \longrightarrow \quad$ & $\norm{\MA-\MB}$              \\
\verb"$\eigvec{\MA}$"              & $\quad \longrightarrow \quad$ & $\eigvec{\MA}$          
\end{tabular}
\end{table}

%$a \yields b$
%
%$a \shallbe b$
%
%$a \definedas b$
%
%$a \isapproximately b$

%$\e^x$
%
%$\ld x$
%
%$\si x$
%
%$\scha x$
%
%$\erf x$

%$\Prob{x\given y}$
%
%$\Erw{X}$
%
%$\Var{X}$
%
%$\Cov{X, Y}$

%\verb"X"              & $\quad \longrightarrow \quad$ & X              \\
%\verb"X"              & $\quad \longrightarrow \quad$ & X              \\





\subsubsection{Fractions}

The built-in LaTeX macro \verb"\frac" tends to decrease the font size
of the numerator and the denominator. This often leads to bad readability.
The macro \verb"\fracds{..}{..}" alleviates this problem:

\verb"$\fracds{1+e^y}{1-e^y{+1}}$" $\qquad \longrightarrow \qquad \fracds{1+e^y}{1-e^y{+1}}$

\clearpage

\section{Image macros}


\image{images/image1}{10cm}{This is image test 1}{img:image1}

\imagesc{images/image1}{10cm}{Small caption}{This is image test 2}{img:image2}

\imagenocap{images/image1}{10cm}

\imagenofloat{images/image1}{10cm}

\doubleimage{images/image1}{5cm}{This is image test 3}{img:image3}{images/image2}{5cm}{This is image test 4}{img:image4}

\newcommand{\testseven}{\dimage{images/image1}{5.5cm}{This is image test 7}{img:image7}}
\newcommand{\testeight}{\dimage{images/image2}{5.5cm}{This is image test 8}{img:image8}}
\doubleimageparts{\testseven}{\testeight}

\doubleimagenocap{images/image1}{5.5cm}{images/image2}{5.5cm}

\doubleimagenofloat{images/image1}{5.5cm}{images/image2}{5.5cm}

\doubleimageonecap{images/image1}{5.5cm}{images/image2}{5.5cm}{These are 2 images}{img:double1}


\doubleimageonecapsc{images/image1}{5.5cm}{images/image2}{5.5cm}{Short caption 10}{These are 2 images}{img:double2}


\ImageAndText[11cm]{images/image1}{5cm}{The image to the left shows the Eelume snake robot mid operation}{5cm}

\TextAndImage[11cm]{The image to the right shows the Eelume snake robot mid operation}{5cm}{images/image1}{5cm}

\end{document}
