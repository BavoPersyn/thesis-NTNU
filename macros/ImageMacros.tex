%=====================================================
%
%       BildMacros.tex
%
%======================================================
%
%  These are macros which are related to representing figures in LaTeX documents
%  in a simple and standardized way.
%   The explanation texts are still largely in German (sorry for that!),
%   but will be translated as soon as possible.
%
%  Source: Rudolf Mester
%=========================================================

%---------------------------------------------------------------------------
%Dependencies
\usepackage{tcolorbox}

%---------------------------------------------------------------------------
%Variables
\newlength{\CVGMBildEins}
\newlength{\CVGMBildZwei}
\newlength{\CVGMBildTotal}
\newlength{\CVGMMinipageEins}
\newlength{\CVGMMinipageZwei}

%---------------------------------------------------------------------------
%   SECTION 1:  commands for text mode
%---------------------------------------------------------------------------
%
% ----- Bilder
%
% Mit den folgenden beiden Befehlen wird eingestellt, ob Breitenangaben in
% den Bildbefehlen absolut (z.B. "7.5cm") oder relativ zur \linewidth
% interpretiert werden (z.B. "0.3"; \linewidth wird automatisch erg\"{a}nzt).
% Voreingestellt ist der absolute Modus.
%
% Selbstverst\"{a}ndlich ist es noch m\"{o}glich, im Absolutmodus durch Angabe
% von \linewidth im Aufruf eine relative Angabe zu machen; durch die
% automatische Erg\"{a}nzung ist es dagegen im Relativmodus nicht m\"{o}glich,
% absolute Angaben zu machen; erst ist der Modus zu wechseln.
%
\newcommand{\absolutemeasures}{\gdef\cvgmode{}}
\newcommand{\relativemeasures}{\gdef\cvgmode{\linewidth}}
\absolutemeasures % default
%
% Grunds\"{a}tzliches:
%
% F\"{u}r 1 oder 2 Bilder gibt es je drei Befehle:
%  - \(doppel)bild : Bild(er) mit Bildunterschrift(en) und Label
%  - \(doppel)bildnocap : ohne Unterschriften, als Flie{\ss}objekt
%  - \(doppel)bildnofloat : keine Unterschriften, kein Flie{\ss}objekt;
%    Bilder "hier und jetzt" im Text, evtl. mit "underfull vbox"
%    auf der n\"{a}chsten Seite ganz oben
%
% Zus\"{a}tzlich gibt es einen Befehl f\"{u}r zwei Bilder mit gemeinsamer
% Bildunterschrift: \doppelbildonecap
%
% Weiterhin gibt es noch Makros f\"{u}r ein Bild mit Text daneben, und zwar
% entweder links Bild und rechts Text oder umgekehrt.
%
%
% Schalter:
%
% Es gibt zwei Schalter (boolean Variablen), die das Verhalten steuern:
%  - \showfilenamestrue  bzw.  \showfilenamesfalse  gibt an, ob hinter
%    der Bildunterschrift der Dateiname erscheinen soll. Dies kann f\"{u}r
%    Debugging-Zwecke hilfreich sein.
%  - \emptypicstrue  bzw.  \emptypicsfalse  dient dazu, noch nicht
%    exitierende Zeichnungen schon einzubinden mit allen Referenzen.
%    Der Dateiname erscheint in einem Kasten (H\"{o}he = 3/4 der angegebenen
%    Bildbreite). Anstelle des Dateinamens kann auch eine kurze
%    Bildbeschreibung angegeben werden, die dann anstelle des Dateinamens
%    angezeigt wird. Beide Schalter gleichzeitig einzuschalten macht
%    aber dann nat\"{u}rlich nicht mehr viel Sinn.
%
\newif\ifshowfilenames\showfilenamesfalse
\def\showfilename#1{\ifshowfilenames\fbox{\textbf{ #1}}\fi}
\newif\ifemptypics\emptypicsfalse
\def\myincludegraphics#1#2{%
    \ifemptypics%
        \@tempdima=#2\cvgmode%
        \fbox{\parbox[b][.75\@tempdima][c]{\@tempdima}{#1}}%
    \else%
        \if@noscale%
	    \includegraphics{#1}%
	\else%
        \includegraphics[width=#2\cvgmode]{#1}%
	\fi%
    \fi%
}
\def\MYincludegraphics#1{%
    \ifemptypics%
        \fbox{\parbox[b][.75\linewidth][c]{\linewidth}{#1}}%
    \else%
        \if@noscale%
        \includegraphics{#1}%
	\else%
        \includegraphics[width=\linewidth]{#1}%
	\fi%
    \fi%
}

\newcommand{\image}[4]
{
    \begin{figure}
        \begin{center}
            \myincludegraphics{#1}{#2}
            \small\caption{\label{#4}#3\showfilename{#1}}
        \end{center}
    \end{figure}
}

%----------------- new command: \bildsc ------------------------------------
%
% Makro zum Anzeigen eines Bildes
% \"{U}bergabeparameter (4): 1: Filename
%                        2: Breite des Bildes
%                        3: kurze Bildunterschrift (Abbildungsverz.)
%                        3: Bildunterschrift
%                        4: Label f\"{u}r Bild
%
\newcommand{\imagesc}[5]
{
  \begin{figure}
      \begin{center}
          \myincludegraphics{#1}{#2}
          \small\caption[#3]{\label{#5}#4\showfilename{#1}}
      \end{center}
  \end{figure}
}

%----------------- new command: \bildnocap ---------------------------------
%
% Bild ohne Bildunterschrift (caption)
% \"{U}bergabeparameter (2): 1: Filename
%                        2: Breite des Bildes
%
\newcommand{\imagenocap}[2]{
    \begin{figure}
        \begin{center}
            \myincludegraphics{#1}{#2}
        \end{center}
        \hspace{0pt plus 1fill}\showfilename{#1}\hspace{0pt plus 1fill}
    \end{figure}
}

% das gleiche als nicht-Flie{\ss}objekt; logischerweise ohne Bildunterschrift
\newcommand{\imagenofloat}[2]{
    \begin{center}
        \myincludegraphics{#1}{#2}
    \end{center}
    \hspace{0pt plus 1fill}\showfilename{#1}\hspace{0pt plus 1fill}
}

%----------------- new command: \doppelbild --------------------------------
%Not working, no CVGMBILDEins, CVGMBILDZwei or CVGBILDTotal
%
% 2 Bilder nebeneinander
% \"{U}bergabeparameter (8): 1: Filename erstes Bild
%                        2: Breite des ersten Bildes
%                        3: Bildunterschrift erstes Bild
%                        4: Label f\"{u}r erstes Bild
%                        5-8: genauso f\"{u}r zweites Bild
%
\newcommand{\doubleimage}[8]{
    \if@noscale
	      % Obtain and save the native width of each picture in the length definitions.
	      \settowidth{\CVGMBildEins}{\MYincludegraphics{#1}}
	      \settowidth{\CVGMBildZwei}{\MYincludegraphics{#1}}
	      \setlength{\CVGMBildTotal}{\CVGMBildEins + \CVGMBildZwei}
        \setlength{\CVGMMinipageEins}{\linewidth * \real{0.95} * \ratio{\CVGMBildEins}{\CVGMBildTotal}}
	      \setlength{\CVGMMinipageZwei}{\linewidth * \real{0.95} * \ratio{\CVGMBildZwei}{\CVGMBildTotal}}        
    \else
	      \setlength{\CVGMMinipageEins}{#2\cvgmode}
	      \setlength{\CVGMMinipageZwei}{#6\cvgmode}
    \fi
    \begin{figure}
        \begin{center}
            \begin{minipage}[t]{\CVGMMinipageEins}
                \if@noscale
                    \centering
                \fi
                \MYincludegraphics{#1}
                \small\caption{\label{#4}#3\showfilename{#1}}
            \end{minipage}
            \hfill
            \begin{minipage}[t]{\CVGMMinipageZwei}
                \if@noscale
                    \centering
                \fi
                \MYincludegraphics{#5}
                \small\caption{\label{#8}#7\showfilename{#5}}
            \end{minipage}
        \end{center}
    \end{figure}
}

%--------------------- new command: \dimage --------------------------------
% Used as an image object for the double image functions below
%
% 1 Bild zur Verwendung mit dem \doppelbildparts Befehl
% \"{U}bergabeparameter (4): 1: Filename Bild
%                        2: Breite des Bildes
%                        3: Bildunterschrift Bild
%                        4: Label f\"{u}r Bild
%

\newcommand{\dimage}[4]{
            \begin{minipage}[t]{#2\cvgmode}
                \MYincludegraphics{#1}
                \small\caption{\label{#4}#3\showfilename{#1}}
            \end{minipage}
}

%--------------------- new command: \dbildsc ------------------------------
%
% 1 Bild zur Verwendung mit dem \doppelbildparts Befehl
% \"{U}bergabeparameter (5): 1: Filename Bild
%                        2: Breite des Bildes
%                        3: kurze Bildunterschrift (Abbildungsverz.)
%                        4: Bildunterschrift Bild
%                        5: Label f\"{u}r Bild
%
\newcommand{\dimagesc}[5]{
            \begin{minipage}[t]{#2\cvgmode}
                \MYincludegraphics{#1}
                \small\caption[#3]{\label{#5}#4\showfilename{#1}}
            \end{minipage}
}

%----------------- new command: \doppelbild --------------------------------
%
% 2 Bilder nebeneinander
% \"{U}bergabeparameter (2): 1: \dbild oder \dbildsc (linkes Bild)
%                        2: ebenso (rechtes Bild)
%
\newcommand{\doubleimageparts}[2]{
    \begin{figure}
        \begin{center}
            {#1}
            \hfill
            {#2}
        \end{center}
    \end{figure}
}

%----------------- new command: \doppelbildnocap ---------------------------
%
% 2 Bilder nebeneinander, ohne \"{U}berschrift
% \"{U}bergabeparameter (4): 1: Filename erstes Bild
%                        2: Breite des ersten Bildes
%                        3-4: genauso f\"{u}r zweites Bild
% Der Platz zwischen den beiden Bildern ist durch den Befehl
% \doppelbildsep bestimmt.
% default = \hfill, d.h. auf volle Breite gestreckt; ggf. umdefinieren,
% z.B. mit \renewcommand{\doppelbildsep}{\hspace{.1\linewidth}}, dann
% automatisch zentriert.
\newcommand{\doppelbildsep}{\hfill}
%
\newcommand{\doubleimagenocap}[4]{
    \begin{figure}
        \begin{center}
            \myincludegraphics{#1}{#2}%
            \doppelbildsep%
            \myincludegraphics{#3}{#4}%
            \hspace{0pt plus 1fill}\showfilename{#1}%
            \hspace{0pt plus 2fill}\showfilename{#3}\hspace{0pt plus 1fill}
        \end{center}
    \end{figure}
}

% das gleiche als nicht-Flie{\ss}objekt; logischerweise ohne Bildunterschriften
%
\newcommand{\doubleimagenofloat}[4]{
    \begin{center}
        \myincludegraphics{#1}{#2}
        \doppelbildsep
        \myincludegraphics{#3}{#4}
    \end{center}
    \hspace{0pt plus 1fill}\showfilename{#1}%
    \hspace{0pt plus 2fill}\showfilename{#3}\hspace{0pt plus 1fill}
}


%----------------- new command: \doppelbildonecap -----------------------
%
% 2 Bilder nebeneinander, EINE Unterschrift
% \"{U}bergabeparameter: 1: Filename erstes Bild
%                    2: Breite des ersten Bildes
%                    3: Filename des zweiten Bildes
%                    4: Breite des zweiten Bildes
%                    5: Bildunterschrift
%                    6: Label
%
\newcommand{\doubleimageonecap}[6]{
    \begin{figure}
        \begin{center}
            \myincludegraphics{#1}{#2}
            \doppelbildsep
            \myincludegraphics{#3}{#4}
            \small\caption{\label{#6}#5\showfilename{#1}\showfilename{#3}}
        \end{center}
    \end{figure}
}

%----------------- new command: \doppelbildonecapsc -----------------------
%
% 2 Bilder nebeneinander, EINE Unterschrift
% \"{U}bergabeparameter: 1: Filename erstes Bild
%                    2: Breite des ersten Bildes
%                    3: Filename des zweiten Bildes
%                    4: Breite des zweiten Bildes
%                    5: kurze Bildunterschrift (Abbildungsverz.)
%                    6: Bildunterschrift
%                    7: Label
%
\newcommand{\doubleimageonecapsc}[7]{
    \begin{figure}
        \begin{center}
            \myincludegraphics{#1}{#2}
            \hfill
            \myincludegraphics{#3}{#4}
            \small\caption[#5]{\label{#7}#6\showfilename{#1}\showfilename{#3}}
        \end{center}
    \end{figure}
}


%%%%%%%%%%%%%%%%%% Bild links, Text rechts
%
% [#1]: Gesamtbreite; OPTIONAL!
% #2: Dateiname Bild
% #3: Breite Bild
% #4: Text
% #5: Breite Text
% Beispiel: \BildUndText[.9]{goethe.eps}{.25}{blablablubb}{.6}
%
\newcommand{\ImageAndText}[5][\linewidth]{
  \begin{center}
    \makebox[#1]{
      \if@noscale
	        % Obtain and save the native width of each picture in the length definitions.
	        \settowidth{\CVGMBildEins}{\myincludegraphics{#2}{#3}}
	        \settowidth{\CVGMBildZwei}{\qquad\quad}
          \setlength{\CVGMMinipageEins}{\linewidth - \CVGMBildZwei - \CVGMBildEins}
      \else
	        \setlength{\CVGMMinipageEins}{#5\cvgmode}
      \fi

      \raisebox{-\totalheight}{\myincludegraphics{#2}{#3}}
      \hfill
      \raisebox{-\totalheight}{\begin{minipage}[b]{\CVGMMinipageEins}{#4}
      \end{minipage}}
    }
    \showfilename{#2}
  \end{center}
}

%%%%%%%%%%%%%%%%%% Text links, Bild rechts
%
% [#1]: Gesamtbreite; OPTIONAL!
% #2: Text
% #3: Breite Text
% #4: Dateiname Bild
% #5: Breite Bild
%
\newcommand{\TextAndImage}[5][\linewidth]{
  \begin{center}
    \makebox[#1]{

      \if@noscale
	        % Obtain and save the native width of each picture in the length definitions.
	        \settowidth{\CVGMBildEins}{\myincludegraphics{#4}{#5}}
	        \settowidth{\CVGMBildZwei}{\qquad\quad}
          \setlength{\CVGMMinipageEins}{\linewidth - \CVGMBildZwei - \CVGMBildEins}
      \else
	        \setlength{\CVGMMinipageEins}{#3\cvgmode}
      \fi

      \raisebox{-\totalheight}{\begin{minipage}[b]{\CVGMMinipageEins}{#2}
      \end{minipage}}
      \hfill
      \raisebox{-\totalheight}{\myincludegraphics{#4}{#5}}
    }
    \showfilename{#4}
  \end{center}
}

