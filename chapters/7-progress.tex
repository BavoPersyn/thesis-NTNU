\chapter{Project Journal}

After reading into Homography and Visual Odometry I started experimenting with OpenCV. The first step was writing a little program that fills a buffer with pictures and simply shows them on screen. The images in this FIFO queue will in a later stage be used to detect keypoints en determine the translation-, rotation- and normal vector. This way the movement of a car can be determined.
\newline\newline
I've already done the following:
\begin{itemize}
    \item read all images of a folder
    \item display all images of a folder and go through them step by step (with multiple stepsizes)
    \item read video and convert every frame to a gray-scale image and save it to a folder
    \item display a menu that gives you the choice of what you want to do:
    \begin{itemize}
        \item read video(s)
        \item go through images in a folder
        \item ...
    \end{itemize}
    
\end{itemize}
The menu makes it easy to try out the different functionalities and check if everything works as expected. It's also easy to experiment with OpenCV this way.
The next step is to read images and store them in a buffer as described. This will be the next item added to the menu.